\documentclass[a4paper,twoside,notitlepage,12pt]{article}
\usepackage{parskip}
\usepackage[nottoc,numbib]{tocbibind}
\usepackage{fullpage}
\usepackage{hyperref}
\usepackage{verbatim}
\usepackage{fancyvrb}
\usepackage{tabularx}


\DefineVerbatimEnvironment{verbatim}{Verbatim}{
    frame=single, baselinestretch=1}
\makeatletter
\addto@hook\every@verbatim{\small}
\makeatother

\renewcommand{\bibsection}{\section{\bibname}}
\setcounter{tocdepth}{2}


\begin{document}

\begin{titlepage}
\begin{center}
    {\Large \bfseries Synthesis of Simple While Programs \\ using Answer Set Programming}

    {\large UROP Project Status, 13 July 2015} \\[0.5cm]
    
    {\large Joseph Crowe} \texttt{<jjc311@imperial.ac.uk>}
    
    \url{http://www.doc.ic.ac.uk/~jjc311/while-synth}\\[2cm]

    \tableofcontents
\end{center}
\end{titlepage}

\section{Introduction to the Problem}

The project is concerned with the automatic synthesis of computer programs in a simple 
Turing-complete imperative programming language with integer arithmetic, branching, and
looping, such as the following:

\begin{verbatim}
// Input x, a non-negative integer.
s = 0;
d = 1;
while (d < x): {
    m = x % d;
    if (m < 1) {
        s = s + d;
    }
    d = d + 1;
}
// Output s, the sum of the proper divisors of x.
\end{verbatim}

The details of this language are subject to change as the project progresses, but it is 
conceptually the same as the basic imperative subset common to the C and Python 
programming languages. The syntax and semantics are documented in Chapter 4 of 
\cite{final}, as they were at the time of writing.

The task is to provide a way to automatically synthesise such a program from a 
specification given by the user, consisting of:

\begin{enumerate}
    \item A \emph{language bias} saying what syntactic elements are allowed to occur in 
    the synthesised program: for example, what integer constants and arithmetic 
    operators may be used, or how many \verb|if| and \verb|while| statements may 
    occur, and how they may be structured; and \textbf{one or both} of:

    \item A \emph{list of examples}, saying exactly what the output of the program 
    should be when certain variables are initialised with certain input values. For 
    example, the following, which specifies by example the above 
    sum-of-divisors program:
    
    \begin{tabular}{| r || r | r | r | r | r | r | r | r | r | r | r | r |}
        \hline
        Input $x$  & 0 & 1 & 2 & 3 & 4 & 5 & 6 & 7 & 9 & 10 & 11 & 12 \\
        \hline
        Output $s$ & 0 & 0 & 1 & 1 & 3 & 1 & 6 & 1 & 4 &  7 &  1 & 16 \\
        \hline
    \end{tabular}

    \item A \emph{functional specification} written as logical propositions --- a 
    \emph{precondition} which must hold for the inputs, at the beginning of the 
    program, and a \emph{postcondition} which must hold for the outputs, at the end of 
    the program --- saying exactly what the valid inputs for the program are and 
    what output is expected for each such input. For example, the following, which 
    again specifies $s$ to end as sum of the proper divisors $k$ of $x_0$:
    
    \begin{tabular}{|ll|}
        \hline
        Pre:  & $x > 0,\ x_0 = x$ \\
        Post: & $s = \sum\{k : k|x_0,\ 0<k<x_0\}$ \\
        \hline
    \end{tabular}
    
\end{enumerate}

\section{Summary of Previous Work}

As this research placement is a continuation of an undergraduate final-year individual 
project, there is an existing body of work that is being extended. This work is 
described in more detail in \cite{final}, but a summary of what was produced will be 
given here.

Note: this section will refer to files in the project's git repository as they they 
were at the end of the individual project, which can be seen using the following 
\emph{tag} on GitHub:

\begin{center} 
    \url{https://github.com/JosephCrowe/ic-while-synth/tree/indiv_final/main}
\end{center}

\subsection{Program Simulator}

An Answer Set Program, \verb|run.lp|, which computes execution traces of While 
programs, was written to serve as a component for both \ref{sec:psynx} and 
\ref{sec:pgenx}. Because of concerns relating to the grounding of the program, a 
modified version is used in \verb|learn.lp| for \ref{sec:psynx} when synthesising 
programs (however, it is possible this could be changed to avoid that duplication). For 
details of the how \verb|run.lp| is implemented, refer to Section~4.3 of \cite{final}.

\subsubsection{Input Format}

Firstly, the simulator must be configured with the following ASP constants, to define 
execution limits:

\begin{tabularx}{\textwidth}{|l|X|}
\hline
\verb|#const time_max=TMX.| &
If more than \verb|TMX| instructions are processed in a single execution trace, that 
trace will fail and be considered to have not terminated. \\
\hline
\verb|#const int_min=IMN.| &
If the value of any variable becomes less than \verb|IMN|, the current execution trace 
will fail. \\
\hline
\verb|#const int_max=IMX.| &
If the value of any variables becomes greater than \verb|IMX|, the current execution 
trace will fail. \\
\hline
\end{tabularx}

Secondly, the program itself is encoded by introducing several facts of the form
\begin{verbatim}
line_instr(N, I).
\end{verbatim}
where \verb|N| is a \emph{line number} starting at 1 and increasing by exactly 1 for
each subsequent line, and \verb|I| is an \emph{instruction}, representing a syntactic
element of the program of one of the following forms:

\begin{tabularx}{\textwidth}{|l|X|}
\hline
\verb|set(V, E)| &
Set variable \verb|V| is to the value of expression \verb|E| \\
\hline
\verb|if(B, L)| &
Execute the next \verb|L| lines if and only if boolean guard \verb|B| holds \\
\hline
\verb|while(B, L)| &
For as long as \verb|B| holds, repeatedly execute the next \verb|L| lines, which must 
be immediately followed by \verb|end_while|. \\
\hline
\verb|end_while| &
May only occur \verb|L| lines after a \verb|while(B, L)| instruction. Marks the point 
where execution may return to the corresponding loop header. \\
\hline
\end{tabularx}

\emph{Variables} are simply represented as nullary function symbols, e.g.\ \verb|x|, 
\verb|y|,\ \verb|z|,\ etc, giving the name of the variable. For the syntax of boolean 
guards and expressions, and commentary on why this representation was chosen, refer to 
Section~4.2 of \cite{final}.

\subsubsection{Output Format}

\subsubsection{Invocation}

\subsection{Synthesis of Programs from Examples} \label{sec:psynx}

\subsection{Automatic Generation of Examples} \label{sec:pgenx}

\section{Current Extensions}

\begin{thebibliography}{9}
    \bibitem{final}
        Crowe, J. (2015).
        Synthesis of Simple While Programs using Answer Set Programming (Final Report).
        \url{http://www.doc.ic.ac.uk/~jjc311/while-synth/WS_FinalReport.pdf}
\end{thebibliography}

\end{document}
